\documentclass[UTF8,openany]{ctexbook}

% 论文版面要求:
% 统一按 word 格式A4纸(页面设置按word默认值)编排、打印、制作。
% 正文内容字体为宋体;字号为小4号;字符间距为标准;行距为25磅(约0.88175cm)。

%%%%% ===== 页面设置
\usepackage[a4paper,top=2.54cm,bottom=2.54cm,left=3.17cm,right=3.17cm,%
            ]{geometry}
            
\setlength{\parindent}{2em}
%默认的弹性间距会导致文中某些排版flush的时候,出现大量空白。
\setlength{\parskip}{0.5em} %指定固定段后间距,默认为弹性间距。
\setlength{\intextsep}{10pt} %固定浮浮动体前后间距。


%%%%% =====章节 标题 设置
%%%%% ===== 中英文字体
\setmainfont{Times New Roman}
%\setsansfont{Myriad Pro} % 无衬线字体 sans serif \sffamily
%\setmonofont{Consolas}   % 等宽字体 typewriter \ttfamily
%\newcommand{\Times}{\fontspec{Times New Roman}}
%% 中文字体
%\setCJKmainfont[BoldFont={Microsoft YaHei},ItalicFont={KaiTi}]{NSimSun}
%\setCJKsansfont{Microsoft YaHei}
%\setCJKmonofont{KaiTi}
%\setCJKfamilyfont{STSong}{方正小标宋_GBK}\newcommand{\STSong}{\CJKfamily{STSong}}
\setCJKfamilyfont{songti}{STZhongsong}\newcommand{\STSong}{\CJKfamily{STSong}}

%%%%% ===== 常用宏包
\usepackage{amsmath,amssymb,amsfonts,bm}
\usepackage[amsmath,thref,thmmarks,hyperref]{ntheorem}
\usepackage{graphicx,xcolor,float}
\usepackage{fancyhdr}
\usepackage{tocloft} % 设置目录中的条目间距


\renewcommand\cftdot{\textsubscript{……}}
\renewcommand\cftdotsep{0}

\setlength{\cftbeforechapskip}{1pt}
\renewcommand{\cftchapleader}{\cftdotfill{\cdot}}


\usepackage{booktabs} % toprule, midrule, bottomrule
\usepackage{varwidth} % 可变宽度的 parbox

%%%%% ===== 参考文献与链接
\usepackage[numbers,sort&compress,sectionbib, square]{natbib} %引用上标,禁用连续缩写。
\newcommand{\upcite}[1]{\textsuperscript{\cite{#1}}}


\usepackage[xetex,pagebackref]{hyperref}
\hypersetup{CJKbookmarks=true,colorlinks=true,citecolor=blue,%
            linkcolor=blue,urlcolor=blue,bookmarksnumbered=true,%
	        bookmarksopen=true,breaklinks=true}
	        
	        
	        
\iffalse   % 调试时,可去掉,以用于显示引用位置。
\renewcommand*{\backrefalt}[4]{%
\ifcase #1 No citations.%
\or Cited on page #2.%
\else Cited on pages #2.%
%\else #1 Cited on pages #2.%
\fi
}

\else
\renewcommand*{\backrefalt}[4]{}
\fi

%%%%% ===== 浮动图表的标题
\usepackage[margin=2em,labelsep=space,skip=0.5em,font=normalfont]{caption}
\DeclareCaptionFormat{mycaption}{{\heiti\color{blue} #1}#2{\kaishu #3}}
\captionsetup{format=mycaption,tablewithin=chapter,figurewithin=chapter}%,belowskip=-10pt
%\setlength{\belowcaptionskip}{-10pt}

%%%%%% ===== 浮动图表的比例默认50%以下,否则无法浮动。
\renewcommand\floatpagefraction{.9} %当浮动体小于页面90%时进行直接放置。
\renewcommand\topfraction{.9}  
\renewcommand\bottomfraction{.9}  
\renewcommand\textfraction{.1}



%%%%% ===== 算法
\usepackage{algorithm,algpseudocode}

%%%%% ===== 其他
\usepackage{ulem}
\def\ULthickness{1pt}




%%%%%===== Code Style代码
\usepackage{listings}
\usepackage{color}

\definecolor{dkgreen}{rgb}{0,0.6,0}
\definecolor{gray}{rgb}{0.5,0.5,0.5}
\definecolor{mauve}{rgb}{0.58,0,0.82}

\lstset{
  language=Python,
  xleftmargin = 3em,xrightmargin = 3em, aboveskip = 1em,
  aboveskip=3mm,
  belowskip=3mm,
  showstringspaces=false,
  columns=flexible,
  rulesepcolor= \color{gray},
  frame = ltrb,
  basicstyle={\normalsize\ttfamily},
  numbers=none,
  numberstyle=\tiny\color{gray},
  keywordstyle=\color{blue},
  commentstyle=\color{dkgreen},
  stringstyle=\color{mauve},
  breaklines=true,
  breakatwhitespace=true,
  tabsize=3
}


\newcommand{\mcc}[1]{\multicolumn{1}{c}{\underline{\makebox[10em][c]{#1}}}}
\newcommand{\mce}[1]{\multicolumn{1}{c}{\underline{\makebox[15em][l]{#1}}}}


\pagestyle{fancy}
\fancyhf{}  % 清除以前对页眉页脚的设置

\newcommand{\makeheadrule}{%% 定义页眉与正文间双隔线
    \makebox[0pt][l]{\rule[.7\baselineskip]{\headwidth}{0.3pt}}%0.4
    \rule[0.85\baselineskip]{\headwidth}{1.0pt}\vskip-.8\baselineskip}
\makeatletter
\renewcommand{\headrule}{%
    % {\if@fancyplain\let\headrulewidth\plainheadrulewidth\fi\makeheadrule}}
    {\makeheadrule}}
\makeatother
\renewcommand{\chaptermark}[1]{\markboth{\CTEXthechapter \ #1}{}}
\renewcommand{\sectionmark}[1]{\markright{\thesection \ #1}{}}
%\fancyhead[RO,LE]{{\small\songti\rightmark}}     % 节标题
%\fancyhead[RE]{{\small\songti\leftmark}}      % 章标题
\fancyhead[L]{《中国文化通论》课程作业}
\fancyhead[R]{李鹏达 10225101460}
% \fancyhead[RO,LE]{$\cdot$ {\small\thepage} $\cdot$}
\fancyfoot[C]{{-\thepage-}}
%\fancyfoot[CO,CE]{{\thepage}}

\begin{document}

\begin{center}
    ~\\[0.5em]
    \zihao{2}
    \heiti{\textbf{金狮子}\\[0.5em]}
    \zihao{-4}
    \kaishu{李鹏达}\\[2em]
\end{center}    

大唐显庆二年\footnote{657年,李弘为唐高宗和武则天修建敬爱寺,“显庆二年。孝敬在春宫。为高宗武太后立之。以敬爱寺为名”\cite{bi:thy}。},东都,敬爱寺。

犀利的冷雨敲在头上,溶解了一个漫长的梦。第一次,我睁开双眼,注视着雨雾中的青石板路。雨水顺着我的脊背滑落,一道道金色的溪流与石板上的雨水汇成一片,映出我的身影。我看着自己的影子,却不知道自己是谁。雨滴从我的发尖落下,荡起一圈波纹,将思绪扰乱。我在这里站立多久了?我为何会在这里?

雨势越来越大了,像是要刺进我的肌肤,却只能顺着我金色的毛发流下。我环顾四周,眺望远处,一层神秘的薄纱蒙住了雨中的洛阳城,若隐若现的屋檐、幽深的小巷、紧闭的木门,每一处都写满了未知。恍惚间,我好像忆起了什么,忆起了千锤百炼、黄金披身的痛,忆起了临别时老师傅的微笑。

一阵脚步声将我从回忆中拉回,几个年轻人从雨中匆匆走来,他们身着青色的长衫,头戴方巾,口中还低声念叨着什么,听上去有些熟悉,却又无法理解。我眯起眼睛,试图听清他们的低语,却总觉得这是另一个世界的语言。雨水打湿了他们的长衫,让他们再次加快了脚步,从我的身边匆匆经过,进入我身后的大殿。

一阵诵经声从大殿内传出,与雨声交织在一起,神秘而庄严。大殿内,烛光透过半开的门缝,投射出暖黄色的光芒,映得灰蒙的雨雾不再那般寒冷。两名香客从大殿中走出,站在屋檐下小声地说着话。我集中精神,隐约听到了“太子”“皇后”“孝顺”几个词,更多的声音淹没在了雨中。

这寺庙大抵是太子为了孝敬皇后修建的吧?我一边思索,一边看着雨渐渐变小,任思绪在风中飘荡,想起老师傅在锻造我时每一锤的敲击对我灵魂的雕琢,想起黄金在身上流淌时的灼热。

耳边的雨声与诵经声渐渐弱了,我注视着面前来往的僧人与香客,他们又有着什么愿望呢?我沉思着,试图从他们匆匆的脚步和深沉的眼神中找寻答案。一位老僧人,步履沉稳,面带宁静,他的眼神中透露出一种超脱,仿佛看穿了世间的一切,却又无法言说。一位年轻的香客,面露虔诚,手持香火,眼神中充满了对信仰的执着,仿佛只有这样才能找到心底的那个回答。

我静静地立在那里,注视着他们来来往往,感受着他们从身旁走过时的气息,做着一只金狮子该做的事——旁观、守护、沉默。

我不懂什么佛法,也听不懂他们口中的咒语。我只是看着、听着、感受着,这一切都是我生命中的一部分。雨渐渐停了,天空中露出了一丝阳光,洒在青石板路上,将我的影子拉得很长。

我不由地又有些困倦。\\[0.5em]

大周天授元年\footnote{690年9月,武则天在洛阳登基即位,改国号为周,改元天授。“九月九日壬午,革唐命,改国号为周。改元为天授”\cite{bi:jts}},神都,敬爱寺。

秋风卷起地面上干枯凋零的落叶,飘散在空中,又与地面相遇。凉风拂过我的脸庞,吹过我略有黯淡的毛发。我伫立在这里已经三十余年了。

“你听说了吗?以后大唐要变成大周了!”一个童子一边用竹扫帚扫着落叶,一边小声地向另一个童子说着。

“我刚刚听说,太后登基了,以后要叫圣神皇帝。”

“之前听师父讲《大云经》说过,太后是弥勒菩萨化身下凡,应为天下之主\footnote{“言则天是弥勒下生,作阎浮提主,唐氏合微。”\cite{bi:jts2}}。”

“四处都有祥瑞发生,太后登基是天意啊。”,两个童子兴奋地说着,一边扫着落叶,一边向大殿走去。

秋风继续吹拂,夹杂着些许凉意。我注视着那两个童子,他们的身影渐行渐远,消失在大殿的深处。

敬爱寺的钟声悠然回荡,宣告着一个新时代的来临。\\[1em]

大周圣历二年\footnote{699年},神都,佛授记寺。

一个僧人从石板路的远处走来,他身披袈裟,双目微上扬,目光中透着宁静与智慧。他手持禅杖,步履沉稳,向大殿走去。

走过我身旁时,他停下脚步,注视着我。我也看着他,他却微微一笑,继续向大殿走去。

“贤首国师,您来了。”一个年轻的僧人走上前,向那位老僧行礼。

贤首国师微笑着,拍了拍年轻僧人的肩膀,走进大殿。

“今天,我们继续讲解《华严经》。”贤首国师的声音在大殿中回荡。

“有十佛世界微尘数菩萨摩诃萨所共围绕\cite{bi:hyo}……”他的声音清晰而庄严。

尽管我已立在这里近四十年,却依然觉得这些佛法晦涩难懂,这么多年来,我一直在这里,看着这里的人们来来往往,听着他们的佛法,也只是理解了一点点。

“妙焰海大自在天王,得法界、虚空界寂静方便力解脱门\cite{bi:hyo}……”佛经的声音在耳边环绕,不由地让我有些疲倦。我慢慢闭上了眼睛。\\[1em]

不知是睡了多久。几天?几个月?

大殿依旧环绕着贤首国师讲经的声音,只是殿外不知为何站了许多卫兵,衬得这里更加庄严肃穆。

我望向大殿,殿内除了贤首国师,只有一位女子。

她身着华丽的锦袍,绣以精致的龙凤图案,头戴金冠,嵌以宝石,映着殿内的烛光。她翻着手中的经文,眉头微蹙,似乎在思索着什么。

讲经的声音渐渐停了,贤首国师望向那女子,向她行礼。

“可朕还是不懂,你所讲的‘十重玄门’”,女子抬起头,看着贤首国师。

僧人眉头紧皱,在殿内踱步,烛光下,他的影子在墙上摇曳。

“陛下,您看这殿外的金狮子”,贤首国师站在大殿门口,指着我,说道。

“这是……”,女子抬起头,看向我。

“工匠的制作,让金和狮子相生成了金狮子。金和狮子相是同时存在的,既没有先后顺序,也完全具备了金和狮子各自的特点。这就是‘同时具足相应门’\cite{bi:hy}。”

女子听着贤首国师的解释,眼神中逐渐有了明悟。

我看着他们,也不由地思索起来。

“陛下,如果您用金狮子的眼去包容整个金狮子相,那金狮子会怎么样呢?”贤首国师问。

女子沉思片刻,说道:“金狮子的眼,包容了整个金狮子,那么整个金狮子就纯粹都是眼。”

“正是如此,如果用耳来包容,那么整个金狮子就纯粹都是耳。也就是说,随便选取一个器官,都包含了其他器官和整个狮子的特征。因此,金狮子的眼耳等各不相同,即‘一一皆杂’;同时,每个器官又能代表整个狮子,即‘一一皆纯’。这种纯杂相应,互不妨碍,圆满地具备一切,这就是‘诸藏纯杂具德门’\cite{bi:hy}。”

女子看向我,思考良久后点了点头。

“金和狮子都在金狮子中,各不相同,却又互相容受\cite{bi:hy},这就是——”

“‘一多相容不同门’。”女子说道。

“如果我说,狮子的眼就是耳,耳就是鼻,鼻就是舌,舌就是身,身就是眼\cite{bi:hy},陛下能否理解?”贤首国师又问。

女子起身,走到我身边,注视着我思索着。为何会有这种道理?我甚是不解。

“是因为金狮子的眼、耳、鼻等感觉器官和狮身上的每一根毛,都能以金包容整个金狮子的形象,这样对吗?”女子说道。

“正是如此,这就是‘诸法相即自在门’。”贤首国师微笑道。

我感受着自己的五官,感受着自己的身体,好像也明白了些什么。

“如果从金狮子的形象来看,就只能看到狮子而看不到金;如果从金的视角来看,就只能看到金而看不到狮子\cite{bi:hy}。”

“如果从以上两个视角观察,那狮子和金就都显现也都隐没。这就是‘秘密隐显俱成门’的含义吗?”女子问。

贤首国师点了点头,女子也微笑。

“金狮子的金和形相,可以隐没或显现,既可以是一,也可以是多。它们必定是纯的,也必定是杂的,可以有力显现或无力隐没,既是此又是彼,主角与陪伴互相辉映,理与事同时显现,互相容纳而不妨碍。即使是极微细的事物,也能包容其他极大的事物,这称为‘微细相容安立门’\cite{bi:hy}。”

女子盯着我,沉思片刻,眼神中突然闪过亮光,向贤首国师点头。

“金狮子的眼、耳、四肢关节以及每一根细小的毛中,都各自包含着金狮子的整体。无数根细小的毛中的狮子,又能同时融入到一根毛中。每根毛中都有无数的狮子,而每根毛又携带着无数狮子归还到一根毛中。如此交互涉入,重重无尽,如同帝释天宫殿中的豪华网珠,网线上的珠玉互相连络,珠光闪耀,彼此辉映,层层叠叠,无穷无尽\cite{bi:hy}。”

“这就是‘因陀罗网境界门’。”女子说道。

我看向身上的毛发,甚是不解,试图从毛发中看到自己的全貌,却总是无法看清。我沉思着,为什么说每根毛中都有无数的狮子?我又是如何包含在每根毛中的呢?

思索甚久,仍是不解。

待我回过神时,讲经已接近结束。

“陛下,看来您已经理解了《华严经》的奥秘了。”贤首国师微笑着说。

圣神皇帝站起身说道:“感谢国师的指点,寡人明白了。”

僧人向她行礼,一同走出大殿。

我看着他们离去,心中仍是疑惑。我看着自己的毛发,陷入了长久的沉思。\\[1em]

不知过了多久。

我再次睁开眼睛,心中终于有了明悟。

殿内传来了讲经的声音。

“今天,我们讲解《华严金师子章》……”
















\newpage

\begin{thebibliography}{99}
\bibitem{bi:thy} 
王博.唐会要.卷四十八.上海古籍出版社. 2006

\bibitem{bi:jts}
刘{\CJKfontspec{宋体}{昫}}等.旧唐书.卷六.本纪第六.中华书局. 1975

\bibitem{bi:jts2}
刘{\CJKfontspec{宋体}{昫}}等.旧唐书.卷一百八十七.中华书局. 1975

\bibitem{bi:hyo}
高振农译.星云大师监修.华严经.东方出版社.北京.2016.

\bibitem{bi:hy}
方立天释译.星云大师监修.华严金师子章.东方出版社.北京.2016.
\end{thebibliography}


\end{document}