\documentclass{article}
\usepackage{algorithm}
\usepackage{algpseudocode}
\usepackage{amsmath}

\begin{document}

\title{Exercise 1}
\author{LI, Pengda \\ 10225101460}
\date{}
\maketitle

\noindent 1. Revise Landau’s O and $\Theta$ notation.
\noindent Explain why the bubblesort algorithm for sorting a list of length $n$ is $O(n^2)$. Is it $\Theta(n^2)$?\\

The bubblesort algorithm can be described as follows:

\begin{algorithm}
\caption{Bubble Sort}
\begin{algorithmic}[1]
\Require Array $A$, length $n$
\For{$i = 1$ to $n - 1$}
    \For{$j = 0$ to $n - i - 1$}
        \If{$A[j] > A[j + 1]$}
            \State Swap $A[j]$ and $A[j + 1]$
        \EndIf
    \EndFor
\EndFor
\State \Return Array $A$
\end{algorithmic}
\end{algorithm}

The outer loop runs $n - 1$ times and the inner loop runs $n - i - 1$ times, where $i$ is the index of the outer loop. The total number of comparisons is 

\[
(n - 1) + (n - 2) + \dots + 1 = \frac{n(n - 1)}{2}
\]

Let $f(n) = \frac{n(n-1)}{2}$, then for any $n > 1$, we have 

\[
f(n) \leq n^2
\]

Therefore, by the definition of $O$ notation, the bubblesort algorithm is $O(n^2)$.

The bubblesort algorithm is $\Theta(n^2)$. To prove this, we need to show that the algorithm is both $O(n^2)$ and $\Omega(n^2)$.

Let $f(n) = \frac{n(n-1)}{2}$, then for any $n \geq 3$, we have

\[
f(n) = \frac{1}{3} n^2 = \frac{n^2}{6} = \frac{n}{2} = \frac{1}{6} n(n - 3) \geq 0
\]

That is, $f(n) \geq Cn^2$ for $n \geq 3$ and $C = 1/3$. Therefore, the bubblesort algorithm is $\Omega(n^2)$.

So the bubblesort algorithm is $\Theta(n^2)$.

\noindent 2. Prove that linear search for an element in an unordered list of length $n$ is $\Theta(n)$.\\

The linear search algorithm can be described as follows:

\begin{algorithm}
\caption{Linear Search}
\begin{algorithmic}[1]
\Require Array $A$, length $n$, element $x$
\For{$i = 0$ to $n - 1$}
    \If{$A[i] = x$}
        \State \Return $i$
    \EndIf
\EndFor
\State \Return $-1$
\end{algorithmic}
\end{algorithm}

In the worst case, the element $x$ is at the end of the list or not in the list. The loop runs $n$ times. Therefore, the linear search algorithm is $\Omega(n)$.

In the average case, the element $x$ is in the list and the probability of finding it is $1/n$. The expected number of comparisons is $n/2$. Therefore, the linear search algorithm is $O(n)$.

Therefore, the linear search algorithm is $\Theta(n)$.

\end{document}
