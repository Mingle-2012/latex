\documentclass{article}[12pt, a4paper]
\usepackage{amsmath, amssymb}

\title{Exercise 4}
\author{LI, Pengda \\ 10225101460}
\date{}

\linespread{1.2}

\newcounter{problemname}
\newenvironment{problem}{\stepcounter{problemname}\par\noindent\textsc{\arabic{problemname}.}}{\\\par}
\newenvironment{solution}{\par\noindent\textsc{Solution. }}{\\\\\par}

\begin{document}
\maketitle

\begin{problem}
    Show that the matrix \( H_n \) is unitary and that quantum initialisation (1) is given by \( H_n \delta_0 \).
\end{problem}

\begin{solution}
    The matrix \( H_n \) is defined as
    \[
    H_{2^n} = 2^{-n/2} \begin{pmatrix} 1 & 1 \\ 1 & -1 \end{pmatrix}^{\otimes n}.
    \]
    We can prove that \( H_n \) is unitary by reduction.
    When $n=2$,
    $$
    H_2 = 2^{-1/2} \begin{pmatrix} 1 & 1 \\ 1 & -1 \end{pmatrix}, 
    H_2^{\dagger} = 2^{-1/2} \begin{pmatrix} 1 & 1 \\ 1 & -1 \end{pmatrix}。
    $$
    Then
    $$
    H_2 H_2^{\dagger} = H_2^{\dagger} H_2 = 2^{-1} \begin{pmatrix} 1 & 1 \\ 1 & -1 \end{pmatrix} \begin{pmatrix} 1 & 1 \\ 1 & -1 \end{pmatrix} = 2^{-1} \begin{pmatrix} 2 & 0 \\ 0 & 2 \end{pmatrix} = I_2.
    $$
    Thus, \( H_2 \) is unitary.
    Suppose that \( H_n \) is unitary, then
    $$
    H_{2n} = H_n \otimes H_2 = 2^{-1/2}\begin{pmatrix}
        H_n & H_n \\
        H_n & -H_n
    \end{pmatrix},
    H_{2n}^{\dagger} = 2^{-1/2}\begin{pmatrix}
        H_n^{\dagger} & H_n^{\dagger} \\
        H_n^{\dagger} & -H_n^{\dagger}
    \end{pmatrix}.
    $$
    $$
    \begin{aligned}
        H_{2n} H_{2n}^{\dagger} = H_{2n}^{\dagger} H_{2n} &= 2^{-1}\begin{pmatrix}
            H_n & H_n \\
            H_n & -H_n
        \end{pmatrix} \begin{pmatrix}
            H_n^{\dagger} & H_n^{\dagger} \\
            H_n^{\dagger} & -H_n^{\dagger}
        \end{pmatrix} \\
        &= 2^{-1}\begin{pmatrix}
            2H_n H_n^{\dagger} & 0 \\
            0 & -2H_n H_n^{\dagger}
        \end{pmatrix} \\
        &= 2^{-1}\begin{pmatrix}
            2I_n & 0 \\
            0 & -2I_n
        \end{pmatrix} \\
        &= I_{2n}.
    \end{aligned}
    $$
    Thus, \( H_{2n} \) is unitary.
    Therefore, \( H_n \) is unitary.

    Then we prove that quantum initialisation (1) is given by \( H_n \delta_0 \).
    
    For \( n = 2 \),
    \[
    H_2 \delta_0 = 2^{-1/2} \begin{pmatrix} 1 & 1 \\ 1 & -1 \end{pmatrix} \begin{pmatrix} 1 \\ 0  \end{pmatrix} = 2^{-1/2} \begin{pmatrix} 1 \\ 1 \end{pmatrix}.
    \]

    Suppose that $H_{2^n} \delta_0 = 2^{-n/2} \begin{pmatrix} 1 \\ 1 \\ \vdots \\ 1 \end{pmatrix}$ holds, then
    \[
    H_{2^{n+1}} \delta_0 = 2^{-1/2} \begin{pmatrix}
        H_{2^n} & H_{2^n} \\
        H_{2^n} & -H_{2^n}
    \end{pmatrix} \begin{pmatrix} 1 \\ 0 \\ \vdots \\ 0 \end{pmatrix} = 2^{-1/2} \begin{pmatrix} H_{2^n} \\ H_{2^n} \end{pmatrix} = 2^{-(n+1)/2} \begin{pmatrix} 1 \\ 1 \\ \vdots \\ 1 \end{pmatrix},
    \]
    which is the initialization of a $(n+1)$-qubit quantum system.

    Thus, quantum initialization (1) is given by \( H_n \delta_0 \).
\end{solution}

\begin{problem}
    Check that the matrix \( \Phi_{\phi} \) for phase shift is unitary.
\end{problem}

\begin{solution}
    $$
    \Phi_{\phi} = \begin{pmatrix}
        1 & 0 \\
        0 & \mathrm{e}^{\mathrm{i}\phi}
    \end{pmatrix},
    \Phi_{\phi}^{\dagger} = \begin{pmatrix}
        1 & 0 \\
        0 & \mathrm{e}^{-\mathrm{i}\phi}
    \end{pmatrix}.
    $$
    Then we have
    $$
    \begin{aligned}
        \Phi_{\phi} \Phi_{\phi}^{\dagger} &= \begin{pmatrix}
            1 & 0 \\
            0 & \mathrm{e}^{\mathrm{i}\phi}
        \end{pmatrix} \begin{pmatrix}
            1 & 0 \\
            0 & \mathrm{e}^{-\mathrm{i}\phi}
        \end{pmatrix} \\
        &= \begin{pmatrix}
            1 & 0 \\
            0 & \mathrm{e}^{\mathrm{i}\phi} \mathrm{e}^{-\mathrm{i}\phi}
        \end{pmatrix} \\
        &= \begin{pmatrix}
            1 & 0 \\
            0 & 1
        \end{pmatrix} \\
        &= I_2, \\
    \end{aligned}
    $$
    and
    $$
    \begin{aligned}
        \Phi_{\phi}^{\dagger} \Phi_{\phi} &= \begin{pmatrix}
            1 & 0 \\
            0 & \mathrm{e}^{-\mathrm{i}\phi}
        \end{pmatrix} \begin{pmatrix}
            1 & 0 \\
            0 & \mathrm{e}^{\mathrm{i}\phi}
        \end{pmatrix} \\
        &= \begin{pmatrix}
            1 & 0 \\
            0 & \mathrm{e}^{-\mathrm{i}\phi} \mathrm{e}^{\mathrm{i}\phi}
        \end{pmatrix} \\
        &= \begin{pmatrix}
            1 & 0 \\
            0 & 1
        \end{pmatrix} \\
        &= I_2.
    \end{aligned}
    $$

    Thus, the matrix \( \Phi_{\phi} \) for phase shift is unitary.
\end{solution}

\begin{problem}
    Recall the operator \( T_f \). Identify \( T_{\oplus} \), for exclusive or, and show that it is unitary.
\end{problem}

\begin{solution}
    $$
    T_{\oplus} \rightsquigarrow \begin{pmatrix}
        1 & 0 & 0 & 0 \\
        0 & -1 & 0 & 0 \\
        0 & 0 & -1 & 0 \\
        0 & 0 & 0 & 1
    \end{pmatrix},
    T_{\oplus}^{\dagger} = \begin{pmatrix}
        1 & 0 & 0 & 0 \\
        0 & -1 & 0 & 0 \\
        0 & 0 & -1 & 0 \\
        0 & 0 & 0 & 1
    \end{pmatrix}.
    $$
    Then
    $$
    \begin{aligned}
        T_{\oplus} T_{\oplus}^{\dagger} = T_{\oplus}^{\dagger} T_{\oplus} &= \begin{pmatrix}
            1 & 0 & 0 & 0 \\
            0 & -1 & 0 & 0 \\
            0 & 0 & -1 & 0 \\
            0 & 0 & 0 & 1
        \end{pmatrix} \begin{pmatrix}
            1 & 0 & 0 & 0 \\
            0 & -1 & 0 & 0 \\
            0 & 0 & -1 & 0 \\
            0 & 0 & 0 & 1
        \end{pmatrix} \\
        &= \begin{pmatrix}
            1 & 0 & 0 & 0 \\
            0 & 1 & 0 & 0 \\
            0 & 0 & 1 & 0 \\
            0 & 0 & 0 & 1
        \end{pmatrix} \\
        &= I_4,
    \end{aligned}
    $$

    So the operator \( T_{\oplus} \) is unitary.
\end{solution}

\begin{problem}
    What is the result of including in the program for the random bit, evolution by
    \[
    2^{-1/2} \begin{pmatrix} 1 & 1 \\ 1 & -1 \end{pmatrix}?
    \]
\end{problem}

\begin{solution}
    It turns a normal state into a quantum state with equal probability of being in the 0 state and the 1 state since
    \[
    2^{-1/2} \begin{pmatrix} 1 & 1 \\ 1 & -1 \end{pmatrix} \begin{pmatrix} 1 \\ 0 \end{pmatrix} = 2^{-1/2} \begin{pmatrix} 1 \\ 1 \end{pmatrix},
    \]
    \[
    2^{-1/2} \begin{pmatrix} 1 & 1 \\ 1 & -1 \end{pmatrix} \begin{pmatrix} 0 \\ 1 \end{pmatrix} = 2^{-1/2} \begin{pmatrix} 1 \\ -1 \end{pmatrix}.
    \]
\end{solution}
\end{document}